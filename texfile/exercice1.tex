\documentclass[20pt]{article}
\usepackage[utf8]{inputenc}
\usepackage{hyperref}


\title{Exercice Chapitre 1}
\author{Spero TESSY \\ Stephene WANTCHEKON}
\date{\today}

\begin{document}

\maketitle

\begin{enumerate}
    \item Utilisez le web scraping pour collecter la liste des médicaments prescrits à partir de : \\ \url{https://www.doctissimo.fr/asp/medicaments/les-medicaments-les-plus-prescrits.htm}
    
    \item Créez un dictionnaire data = \{ 'Medicament' : values \}, avec values la liste des médicaments recueillis sur le site internet.
    
    \item Créez un fichier JSON et stockez la base de données dans ce fichier JSON.
    
    \item Utilisez le module sqlite3 et créez une base de données nommée "medicaments".
    
    \item Créez une table "medicaments" avec une clé primaire "Id" de type integer et autoincrement, et une clé secondaire "prescrits".
    
    \item Utilisez une boucle pour injecter tous les médicaments dans la base de données.
\end{enumerate}

\section{Pour aller plus loin}

Utilisez ce format de lien pour accéder à la page URL de chaque médicament :

Exemple : \url{https://www.doctissimo.fr/medicament-<<DOLIPRANE>>.htm}

Remplacez <<DOLIPRANE>> par le nom du médicament.

Pour chaque médicament, collectez les informations suivantes :

\begin{itemize}
    \item L'état du médicament (si le médicament est toujours sur le marché)
    \item Le prix de vente
    \item Le taux de remboursement social
    \item La classe thérapeutique
    \item Le principe actif
    \item Les effets indésirables
    \item La liste des médicaments similaires
    \item Les précautions d'emploi
    \item Les contre-indications
\end{itemize}

Stockez toutes ces données dans une base de données SQL de façon bien structurée.




\section{Finalisation du projet}

Pour conclure ce projet :

\begin{enumerate}
    \item Créez un nouveau repository sur GitHub.
    \item Initialisez Git dans votre dossier de travail local.
    \item Ajoutez tous les fichiers créés (codes, bases de données, etc.) à votre repository local.
    \item Effectuez un commit avec un message descriptif.
    \item Poussez tous les fichiers vers votre repository GitHub distant.
\end{enumerate}

Assurez-vous d'inclure tous les fichiers pertinents, y compris les scripts Python, les fichiers JSON, et la base de données SQLite.
Ce sera le lien que vous m'envoierez pour la correction. 
Bonsoir et bonne reflexion.


\end{document}
